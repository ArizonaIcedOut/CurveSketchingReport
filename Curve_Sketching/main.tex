%% %%%%%%%%%%%%%%%%%%%%%%%%%%%%%%%%%%%%%%%%%%%%%%%%
%% Problem Set/Assignment Template to be used by the
%% Food and Resource Economics Department - IFAS
%% University of Florida's graduates.
%% %%%%%%%%%%%%%%%%%%%%%%%%%%%%%%%%%%%%%%%%%%%%%%%%
%% Version 1.0 - November 2019
%% %%%%%%%%%%%%%%%%%%%%%%%%%%%%%%%%%%%%%%%%%%%%%%%%
%% Ariel Soto-Caro
%%  - asotocaro@ufl.edu
%%  - arielsotocaro@gmail.com
%% %%%%%%%%%%%%%%%%%%%%%%%%%%%%%%%%%%%%%%%%%%%%%%%%

\documentclass[12pt]{article}
\usepackage{design_ASC}
\usepackage{amsmath}

\setlength\parindent{0pt} %% Do not touch this
\usepackage[margin=1in]{geometry}

%% -----------------------------
%% TITLE
%% -----------------------------
\title{Calculus Curve Sketching Assignment} %% Assignment Title

\author{Eric Pu\\ %% Student name
MCV4U\\ %% Code and course name
}

%% %%%%%%%%%%%%%%%%%%%%%%%%%
\begin{document}
\setlength{\droptitle}{-5em}    
%% %%%%%%%%%%%%%%%%%%%%%%%%%
\maketitle

% --------------------------
% Start here
% --------------------------

% %%%%%%%%%%%%%%%%%%%
\section*{The Function}
% %%%%%%%%%%%%%%%%%%%

\begin{align*}
f(x) & = \dfrac{x^2-6x+9}{x^2+x-2} \\& = \dfrac{(x-3)^2}{(x+2)(x-1)}
\end{align*}


% %%%%%%%%%%%%%%%%%%%
\subsection*{X-Intercepts}

{\bfseries \[0 = (x-3)^2\] \n \[x=3\]}

$f(x)$ has one x-intercept at $x = 3$.

% %%%%%%%%%%%%%%%%%%%

\subsection*{Y-Intercepts}

\begin{align*}
f(0) &= \dfrac{(0)^2-6(0)+9}{(0)^2+(0)-2} \\&= -\dfrac{9}{2}
\end{align*}

The intercept of $f(x)$ is $y = -\dfrac{9}{2}$.

\subsection*{Horizontal Asymptotes}

\begin{align*}
y &= \frac{\deg{numerator}}{\deg{denominator}} \\&= \dfrac{2}{2} \\&=1
\end{align*}

Since the degree of the numerator is equal to the degree of the denominator, $f(x)$ has a horizontal asymptote at $y = 1$.

\subsection*{Vertical Asymptotes}

{\bfseries \[0 = (x+2)(x-1)\] \n \[x = -2, 1\]}

$f(x)$ has two vertical asymptotes, at $x = -2, 1$.

\subsection*{Behaviour Around Asymptotes}

To determine the behaviour of the function, we must first solve for $f(x)$ as $x$ approaches our vertical asymptotes and infinity.

\begin{table}[h]
\centering
\begin{tabular}{|c|c|lll}
\cline{1-2}
$x$    & $f(x)$   &  &  &  \\ \cline{1-2}
-1.9 & -82.79 &  &  &  \\ \cline{1-2}
-2.1 & 83.90  &  &  &  \\ \cline{1-2}
0.9  & -15.21 &  &  &  \\ \cline{1-2}
1.1  & 11.65  &  &  &  \\ \cline{1-2}
100  & 0.93   &  &  &  \\ \cline{1-2}
-100 & 1.07   &  &  &  \\ \cline{1-2}
\end{tabular}
\label{tab:my-table}
\end{table}

With this, we now have the limits for $f(x)$ as $x$ approaches the asymptotes. We also know that $f(x)$ is below the horizontal asymptote as $x$ approaches infinity, and it is above the H.A. when approaching negative infinity.

\[\lim_{x \to -2^-} f(x) = \infty\] 
\[\lim_{x \to -2^+} f(x) = -\infty\]
\[\lim_{x \to 1^-} f(x) = -\infty\]
\[\lim_{x \to 1^+} f(x) = \infty\]
\[\lim_{x \to +\infty} f(x) = 1\]
\[\lim_{x \to -\infty} f(x) = 1\]

\section*{First Derivative}

\begin{align*}
f'(x) & = \dfrac{(2x-6)(x^2+x-2)-(2x+1)(x^2-6x+9)}{(x^2+x-2)^2}
\\&= \dfrac{(2x^3-4x^2-10x+12) - (2x^3-11x^2+12x+9)}{(x^2+x-2)^2}
\\&= \dfrac{7x^2-22x+3}{(x^2+x-2)^2}
\end{align*}

\subsection*{X-Intercepts}

\begin{align*}
x &= \dfrac{-(-22) \pm \sqrt{(-22)^2-4(7)(3)}}{2(7)}
\\&= \dfrac{22 \pm \sqrt{400}}{14}
\end{align*}

\[x \doteq 0.143, 3\]

The first derivative of $f(x)$ has two x-intercepts, at $x \doteq 0.143$ and $x = 3$.

\subsection*{Vertical Asymptotes}
\[0 = (x+2)(x-1)\]
\[x = -2, 1\]

$f'(x)$ has two vertical asymptotes, at $x = -2$ and $x = 1$.

\subsection*{Intervals of Increasing/Decreasing}

Using the x-intercepts and vertical asymptotes of the first derivative, the local maximums/minimums and intervals of increasing/decreasing can be found using an interval table.

\begin{table}[h]
\caption{\uparrow = Increasing, \downarrow = Decreasing, und = Undefined\\}
\begin{tabular}{l|c|c|c|c|c|c|c|c|c|}
\cline{2-10}
                            & (-\infty, -2) & x = -2 & (-2, 0.143) & x = 0.143     & (0.143, 1) & x = 1 & (1, 3) & x = 3         & (3, \infty) \\ \hline
\multicolumn{1}{|l|}{$f'(x)$} & +        & und    & +           & 0             & -          & und   & -      & 0             & +      \\ \hline
\multicolumn{1}{|l|}{$f(x)$}  & \uparrow        & und    & \uparrow           & Local Max & \downarrow          & und   & \downarrow      & Local Min & \uparrow      \\ \hline
\end{tabular}
\end{table}

$f(x)$ is increasing when $x \in  (-\infty, 0.143)  \;\cup\;  (3, \infty)$. It is decreasing when $x  \in  (0.143, 3)$.


\subsection*{Local Max/Min}

The specific y-values at the maximum and minimum points can be found by substituting the x-values into the original equation. 

\begin{align*}
f(0.143) &= \dfrac{(0.143)^2-6(0.143)+9}{(0.143)^2+(0.143)-2}
\\&\doteq -4.444 
\end{align*}

\begin{align*}
f(3) &= \dfrac{(3)^2-6(3)+9}{(3)^2+(3)-2}
\\&= 0
\end{align*}

Now, we know that $f(x)$ has a local maximum at $(0.143, -4.444)$ and a local minimum at $(3, 0)$.

\section*{Second Derivative}

\begin{align*}
f''(x) &= \dfrac{(14x-22)(x^2+x-2)^2-(7x^2-22x+3)(\frac{\mathrm{dy} }{\mathrm{d} x}[(x^2+x-2)^2])}{(x^2+x-2)^4}
\\&= \dfrac{(14x-22)(x^2+x-2)^2-2(7x^2-22x+3)(x^2+x-2)(2x+1)}{(x^2+x-2)^4}
\\&= \dfrac{(14x-22)(x^2+x-2)-2(7x^2-22x+3)(2x+1)}{(x^2+x-2)^3}
\\&= \dfrac{-14x^3+66x^2-18x+38}{(x^2+x-2)^3}
\\&= -\dfrac{2(7x^3-33x^2+9x-19)}{(x^2+x-2)^3}
\end{align*}

\subsection*{X-Intercepts}

Unfortunately, my attempts to use rational zero theorem to solve for the roots of the numerator failed. Testing all possible values of $\dfrac{b}{a}$, none of them resulted in 0 when substituted into $f''(x)$. Thus, I had to use my calculator's cubic solving functionality to find the x-intercept of the second derivative. Using this method, I found that $f''(x)$ has one x-intercept, at $x \doteq 4.563$.

\subsection*{Vertical Asymptotes}

\begin{align*}
0 &= (x^2+x-2)^3
\\&= (x-1)^3(x+2)^3
\end{align*}

\[x = 1, -2\]

$f''(x)$ has two vertical asymptotes, at $x = -2$ and $x = 1$.

\subsection*{Intervals of Concavity}

\begin{table}[h]
\centering
\caption{\uparrow\; = Concave \;Up,\; \downarrow \;= \;Concave Down, PoI = Point \;of \;Inflection\n}
\begin{tabular}{l|c|c|c|c|c|c|c|}
\cline{2-8}
                            & \begin{tabular}[c]{@{}c@{}}(-\infty,  -2)\end{tabular} & x = -2 & \begin{tabular}[c]{@{}c@{}}(-2,  1)\end{tabular} & x = 1 & (1, 4.563) & x = 4.563 & (4.563, \infty) \\ \hline
\multicolumn{1}{|l|}{$f''(x)$} & +                                                   & und    & -                                                  & 0     & +          & 0         & -              \\ \hline
\multicolumn{1}{|l|}{$f(x)$}  & \uparrow                                                   & PoI    & \downarrow                                                  & PoI   & \uparrow          & PoI       & \downarrow              \\ \hline
\end{tabular}
\end{table}

According to the interval table, $f(x)$ is concave up when $x  \in  (-\infty, -2)  \;\cup\;  (1, 4.563)$. $f(x)$ is concave down when $x  \in  (-2, 1) \; \cup \; (4.563, \infty).$ It has points of inflection at its vertical asymptotes and x-intercept, at $x = -2$, $x = 1$, and $x \doteq 4.563$.
\\\\
We now have ample information to graph our function.

\end{document}